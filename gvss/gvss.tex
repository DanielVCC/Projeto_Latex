\documentclass{article}
\usepackage[utf8]{inputenc}

\title{IF738-Redes de Computadores}
\author{Gislayne Vitorino dos Santos Silva }
\date{\vspace{-5ex}}

\usepackage{natbib}
\usepackage{graphicx}

\begin{document}

\maketitle

\section{Introdução}
Esta disciplina visa, de forma abrangente ao ensino e estudos da parte introdutória de redes de computadores, que se trata de uma estrutura que interliga dispositivos permitindo a comunicação e troca de informações entre eles. No CIn os tópicos tratados pela disciplina são, em sua maioria, escolhidos pelos discentes juntamente ao professor vigente.
\cite{primeira, segunda}
Porém há alguns tópicos obrigatórios, veja os principais a seguir:
\begin{itemize}
   \item Modelo de Referência OSI: O Modelo de referência OSI é o modelo fundamental para comunicações em rede, por meio de tal modelo ocorre a padronização e esta possibilita a comunicação entre máquinas distintas, amenizando os conflitos na hora da troca de informações através dos protocolos. Ela possui várias camadas que são devidamente analisadas ao longo do período;
   \cite{terceira, sexta}
   \item Tipos de redes: Há diversos tipos de redes, algumas são notoriamente mais conhecidas como as LAN e WAN, porém não se resumem apenas a essas. Diferentes modelos de redes de computadores existem para as diversas necessidades que venham a aparecer. Algumas delas possuem um enfoque maior no curso, tais como as: Redes sem fio e Ópticas.
   \cite{setima}
 \end{itemize}


\begin{figure}[h!]
\centering
\includegraphics[scale=0.3]{networking2}
\caption{Redes de Computadores}
\cite{nona}
\label{fig: networking2}
\end{figure}

\section{Relevância}
"Uma rede é dois ou mais computadores ligados entre si para compartilhar informações e arquivos entre eles."
\cite{quarta}
Apenas esta frase demonstra a tamanha importância e influência de redes de computadores, esta possibilita a comunicação entre máquinas e dispositivos variados. Além de permitir o acesso a Internet que cada vez mais se populariza no mundo moderno. A quantidade de benefícios é inúmera, todavia os malefícios também existem, tais como: invasão de hackers e vírus. Com tantos perigos surgindo num ambiente comum a diversas pessoas, é notória a necessidade de um estudo aprofundado e coerente sobre essa disciplina, por tratar-se da base. Clarificando a necessidade e importância desta na base curricular.


\section{Relações interdisciplinares}
\begin{tabular}{|c|l|}
\hline
Outras disciplinas  & \multicolumn{1}{c|}{Relações}           \\ \hline
\begin{tabular}[c]{@{}c@{}}IF685 \\  Gerenciamento de \\ Dados e Informação\end{tabular} & \begin{tabular}[c]{@{}l@{}}Estuda o gerenciamento de diversas redes, in-\\ cluindo a de computadores. Tendo,porém, um\\ grande enfoque nos protocolos, no que tange a\\ área de redes de computadores, por exemplo\\  TCP e IP.
\cite{oitava}
\end{tabular}                       \\ \hline
\begin{tabular}[c]{@{}c@{}}IF678\\  Infra-estrutura de \\ Comunicação\end{tabular}       & \begin{tabular}[c]{@{}l@{}}Ambas abordam certos meios utilizados para \\ se comunicar através da computação. Con-\\ sequentemente estas estudam diversos temas\\ em comum, por exemplo: Modelo OSI e a \\ própria Introdução à Redes de Computadores.
\cite{quinta}
\end{tabular} \\ \hline
\end{tabular}

\bibliographystyle{ieeetr}
\bibliography{references}
\end{document}
