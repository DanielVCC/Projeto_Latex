\documentclass[10pt]{article}
\usepackage[utf8]{inputenc}
\usepackage[portuguese]{babel}
\usepackage{float}
\usepackage{natbib}
\usepackage[T1]{fontenc}
\usepackage{indentfirst}
\usepackage{graphicx}
\usepackage{hyperref}


\title{IF673 - Lógica para Computação}
\author{Matheus Rocha}
\date{\vspace{-5ex}}

\begin{document}

\maketitle

\section{Introdução}
A disciplina de Lógica para computação tem como objetivo introduzir o aluno às técnicas do raciocínio dedutivo, utilizando para isso a lógica matemática. Para isso, é utilizado tipos diferentes de expressões, que buscam relacionar conjuntos, que são os elementos de estudo dessa disciplina. Os principais elementos utilizados para o seu aprendizado são: 
\begin{itemize}
    \item \textbf{Teoria dos conjuntos:} É a área da matemática que tem como objetivo estudar os conjuntos e suas relações. Conjuntos são classificados como a junção de elementos, que através de operações, podem ser unidos com elementos de outros conjuntos, ou ser tirada a interseção, etc.
    \item \textbf{Álgebra de Boole:} A Álgebra de Boole é utilizada nessa disciplina para fazer as operações de relações entre conjuntos, que são vistos na teoria dos conjuntos, e, assim, facilitando encontrar a solução de problemas lógicos de forma a economizar tempo e recursos.
    \citep{site4}
\end{itemize}
\begin{figure}[h]
    \centering
    \includegraphics[scale=0.6]{imagem1.png}
    \caption{A imagem \citep{site1} mostra uma tabela verdade, que é usada na lógica para computação.}
    \label{fig:my_label}
\end{figure}
\section{Relevância}
Essa disciplina é importante para o currículo de um cientista da computação pelo fato de que o discente desenvolve suas habilidades de pensamento lógico, necessárias para a criação e o aperfeiçoamento de algoritmos eficazes, assim, reduzindo os custos da criação do mesmo e melhorando o seu entendimento.
\section{Relação com outras disciplinas}
Essa disciplina tem relação com várias outras disciplinas que tem como base o uso da lógica. Algumas dessas disciplinas são:
\begin{table}[h]
    \centering
    \begin{tabular}{|l|l|}
    \hline    Disciplina & Relação \\
    \hline     & Matemática Discreta é uma disciplina  \\
         IF670 - Matemática Discreta & base para o estudo da Lógica para Pro-\\
          & gramação,  aluno no estudo dos conjun-\\
          &tos  e operações com os mesmos.\citep{site5}\\
    \hline & Infraestrutra de Software utiliza Lógica \\
      IF677 -  Infraestrutura de  & para Programação para a decisão de\\
      Software  &  processos a serem executados, diminuin-\\
        & do os bugs no código.\citep{site6}\\
    \hline
    \end{tabular}
    \caption{Relação entre disciplinas \citep{site2} \citep{site3}}
    \label{tab:my_label}
\end{table}

\bibliographystyle{plain}
\bibliography{mcr}
\end{document}
