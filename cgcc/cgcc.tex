\documentclass{article}
\usepackage[utf8]{inputenc}

\title{IF682 - Engenharia de Software e Sistemas}
\author{Charles Gabriel Carvalho Cristovão}
\date{Abril 2019}

\usepackage{natbib}
\usepackage{graphicx}
\usepackage{url}
\usepackage[brazil]{babel}
\usepackage[table,xcdraw]{xcolor}

\begin{document}

\maketitle

\section{Introdução}
\paragraph{}Engenharia de software é uma área da computação voltada para criação, manipulação e manutenção de softwares, os principais propósitos são a elaboração de sistemas eficientes, que visam qualidade, praticidade e produtividade. Além disso a engenharia de software oferece mecanismos para se planejar e gerenciar projetos de desenvolvimento de sistemas computadorizados e que cumpra as necessidades do requisitante. Em resumo, a engenharia de software se dedica as teorias, métodos e ferramentas para o desenvolvimento de um software profissional.\cite{pagina} Alguns elementos importantes são:\newline
\textbf{-Software:} é uma sequência de instruções escritas para serem interpretadas por um computador com o objetivo de executar tarefas específicas. Em um computador, o software é classificado como a parte lógica cuja função é fornecer instruções para o hardware.\citep{slide1}

\begin{figure}[ht]
    \centering
    \includegraphics[scale=0.3]{fluxograma2.png}
    \caption{Modelo espiral do processo de software}\citep{slide2}
    \label{fig:fluxograma2}
\end{figure}
\newpage
\textbf{-Qualidades de um software:} um bom software é determinado por vários conceitos, os principais deles são: qualidade interna e externa do software; execução de um programa funcionalmente correto; confiabiliade, que trata-se da possibilidade de o software trabalhar como desejado; robustez, basicamente a capacidade do software de funcionar em circustâncias não especificadas; desempenho, uso dos recursos computacionais economicamente e portabilidade, que diz respeito a capacidade de executar um software em ambientes divergentes.\citep{slide3}
\section{Relevância}
\paragraph{}A relevância dessa cadeira para um programador é ampla, pois trata-se da aplicação bruta dos conceitos lógicos e de programação aprendidos ao longo do curso. O desenvolvimento prático de criação de softwares desenvolve a capacidade dos alunos de manusear técnicas e ferramentas adequadas para cada situação, analisar e indentificar oportunidades de melhoria e qualidade do software, além de capacitar os programadores a trabalharem em conjunto no desenvolvimento dos mesmos.

\section{Relação com outras disciplinas}

\begin{table}[h]
\begin{tabular}{ll}
\multicolumn{1}{c}{\textbf{Disciplina}}                                                                                                                     & \multicolumn{1}{c}{{\color[HTML]{000000} \textbf{Relação}}}                                                                                                       \\ \hline
\rowcolor[HTML]{32CB00} 
\multicolumn{1}{|l|}{\cellcolor[HTML]{32CB00}{\color[HTML]{000000} \begin{tabular}[c]{@{}l@{}}IF685 - Gerenciamento de dados \\ e informação\end{tabular}}} & \multicolumn{1}{l|}{\cellcolor[HTML]{32CB00}{\color[HTML]{000000} \begin{tabular}[c]{@{}l@{}}Manipulação de dados\\  e controle de informação\end{tabular}}}      \\ \hline
\rowcolor[HTML]{34CDF9} 
\multicolumn{1}{|l|}{\cellcolor[HTML]{34CDF9}{\color[HTML]{000000} IF677 - Infra- estrutura de software}}                                                   & \multicolumn{1}{l|}{\cellcolor[HTML]{34CDF9}{\color[HTML]{000000} \begin{tabular}[c]{@{}l@{}}Funcionamento do sistema \\ computacional\end{tabular}}}             \\ \hline
\rowcolor[HTML]{32CB00} 
\multicolumn{1}{|l|}{\cellcolor[HTML]{32CB00}{\color[HTML]{000000} IF681 - Interfaces usuário e máquina}}                                                   & \multicolumn{1}{l|}{\cellcolor[HTML]{32CB00}{\color[HTML]{000000} \begin{tabular}[c]{@{}l@{}}Criação de interfaces \\ para manipulação do software\end{tabular}}} \\ \hline
\end{tabular}
\end{table}

\bibliographystyle{plain}
\bibliography{references}
\end{document}
